\section {Appendix}
% ------
% R Code
% ------
\subsection{Bitcoin - Time Series Linear Regression}
\begin{minted}{r}
require(jsonlite)

##############################################
# get.polo.url()
#
#   Creates a poloniex api call url
#   Either for training (train=T) or testing (train=F)
#   Testing end is 90 days past training end
#
# @param start, default=1
#   The start of your training dataset
# @param end, default=1
#   The end of your training dataset
# @param pair, default="USDT_BTC"
#   Token pairs, taken from the market list on https://poloniex.com/
# @param period, default=86400
#   Period of dataset in seconds, default=1 day
# @param train, default=TRUE
#   Whether you want the training or testing url
# @returns string
#   API URL for accessing poloniex data
##############################################
get.polo.url <- function(start=1, end=1, pair="USDT_BTC", period=86400, train=TRUE) {
  # Returns OHLC + Volume
  POLO_URL <- "https://poloniex.com/public?command=returnChartData&currencyPair=%s&start=%d&end=%d&period=%i"
  # Returns tick by tick trades (max 50000)
  # POLO_URL2 <- "https://poloniex.com/public?command=returnTradeHistory&currencyPair=%s&start=%d&end=%d"

  START <- as.numeric(as.POSIXct(sprintf("2013-%d-01", start)), tz="GMT")
  END <- as.numeric(as.POSIXct(sprintf("2017-%d-01", end)), tz="GMT")
  PRED_START <- as.numeric(as.POSIXct(sprintf("2017-%d-01", end)), tz="GMT")
  PRED_END <- as.numeric(as.POSIXct(sprintf("2017-%d-01", end)), tz="GMT")+7776000
  # PERIOD_VALUES <- c(300, 7200, 86400)
  # 5min, 2hours, 24hours
  PREDICTION_LENGTH <- 90*86400/PERIOD # days
  if(train) {
    return(sprintf(POLO_URL, pair, START, END, period))
  }
  else {
    return(sprintf(POLO_URL, pair, PRED_START, PRED_END, period))
  }
}

# -------------------------------
# Data Creation and Visualization
# -------------------------------
df <- fromJSON(get.polo.url(start=1, end=1))
pred_df <- fromJSON(get.polo.url(start=1,end=1,train=F))
train <- ts(df[8])
test <- ts(pred_df[8])

plot(log(c(train, test)), type="l", xlab="Time (Days)", ylab="Log of Volume Weighted Average Price", main="VWAP vs Time")
lines(x=seq(length(train)+1, length(train)+length(test)), y=log(test), col="red")

# -----------
# Differences
# -----------
plot(diff(train), ylab="Lagged Difference of VWAP", xlab="Time (Days)", main="diff(VWAP) vs Time")
abline(h=0)

# --------------
# ACFs and PACFs
# --------------
acf(diff(train), lag.max=50, main="ACF of diff(train)")
pacf(diff(train), lag.max=50, main="PACF of diff(train)")
acf(diff(diff(train)), lag.max=50, main="ACF of diff(diff(train))")
pacf(diff(diff(train)), lag.max=50, main="PACF of diff(diff(train))")

# --------------------
# Model and Prediction
# --------------------
ARIMA <- arima(train, order=c(1,1,0), seasonal=list(order=c(1,1,1), period=41))
print(AIC(ARIMA))
print(BIC(ARIMA))
ARIMA.pred <- predict(ARIMA, n.ahead=PREDICTION_LENGTH)
error <- abs((c(test)-ARIMA.pred$pred))
error <- ((error/max(error))*(min(log(test))/max(log(test))))+min(log(test))
plot(log(test), lwd=2, xlab="Time (Days)", ylab="Log of VWAP", main="Log of Bitcoin VWAP\nPrediction compared to Actual Prices")
lines(x=seq(along=test), log(ARIMA.pred$pred), col="blue", lwd=1)
plot(log(ts(df2[8])), type='l', main="Log of Bitcoin VWAP with ARIMA prediction", ylab="Log of VWAP")
lines(x=seq(length(series)+1, length(series)+length(ARIMA.pred$pred)), y=log(ARIMA.pred$pred), col="red")

# ------------------------
# Recurrent Neural Network
# ------------------------
library(rnn)
df <- fromJSON(get.polo.url(start=1, end=1))
X = matrix(1:680, nrow=34)
Y = matrix(df[1:680,8], nrow=34)

# Conform to interval [0,1]
X <- (X-min(X)) / (max(X)-min(X))
Y <- (Y-min(Y)) / (max(Y)-min(Y))

# Transpose
X <- t(X)
Y <- t(Y)

train <- 1:15
test <- 16:20

model <- trainr(Y = Y[train,],
                X = X[train,],
                learningrate = 0.05,
                hidden_dim = c(16,8),
                numepochs = 1000)

Y.pred <- predictr(model, X)

plot(as.vector(t(Y)), col="red", type="l", main="Actual vs Predicted", ylab="Y, Y.pred")
lines(as.vector(t(Y.pred)), type="l", col="blue")
\end{minted}

% ------------
% Python Codes
% ------------
\subsection{\textit{k} and \textit{k}++ means clustering}
\begin{minted}{python}
import numpy as np
import scipy
import matplotlib.pyplot as plt


class Kmeans(object):
    def __init__(self, data_as_txt, k, seed=None):
        self.k = k
        self.data_as_txt = data_as_txt
        self.seed = seed

    def load_dataset(self, name):
        return np.loadtxt(name)

    def euclidian(slef, a, b):
        return np.linalg.norm(a-b)  # magnitude between two points

    def plot(self, dataset, history_centroids, belongs_to):
        colors = ['r', 'g', 'y', 'c']

        fig, ax = plt.subplots()


        for index in range(dataset.shape[0]):
            instances_close = [i for i in range(len(belongs_to)) if belongs_to[i] == index]
            for instance_index in instances_close:
                ax.plot(dataset[instance_index][0], dataset[instance_index][1], (colors[index] + 'o'))

        history_points = []
        for index, centroids in enumerate(history_centroids):
            for inner, item in enumerate(centroids):
                if index == 0:
                    history_points.append(ax.plot(item[0], item[1], 'bo')[0])
                else:
                    history_points[inner].set_data(item[0], item[1])
                    plt.pause(0.8)


    ''' num_instances: number of rows in data (i.e how many points)
        dataset: the data
        return: list of k randomly selected data points in a list: [[x1 y1] [x2 y2] ... [xk yk]]
    '''
    def initialization(self, dataset):

        if self.seed is not None:
            np.random.seed(self.seed)

        num_instances, num_features = dataset.shape  # 45, 2
        return dataset[np.random.choice(num_instances - 1, self.k, replace=False)]
        #return dataset[np.random.randint(0, num_instances - 1, size = self.k)]

    def kmeans(self, k, epsilon=0, distance='euclidian'):
        history_centroids = []
        if distance == 'euclidian':
            dist_method = self.euclidian
        dataset = self.load_dataset(self.data_as_txt)
        # dataset = dataset[:, 0:dataset.shape[1] - 1]
        num_instances, num_features = dataset.shape  # 45, 2


        #prototypes = dataset[np.random.randint(0, num_instances - 1, size=k)]
        prototypes = self.initialization(dataset)
        history_centroids.append(prototypes)
        prototypes_old = np.zeros(prototypes.shape)
        belongs_to = np.zeros((num_instances, 1))
        norm = dist_method(prototypes, prototypes_old)
        iteration = 0

        while norm > epsilon:
            iteration += 1
            norm = dist_method(prototypes, prototypes_old)
            prototypes_old = prototypes
            for index_instance, instance in enumerate(dataset):
                dist_vec = np.zeros((k, 1))
                for index_prototype, prototype in enumerate(prototypes):
                    dist_vec[index_prototype] = dist_method(prototype,
                                                            instance)

                belongs_to[index_instance, 0] = np.argmin(dist_vec)

            tmp_prototypes = np.zeros((k, num_features))

            for index in range(len(prototypes)):
                instances_close = [i for i in range(len(belongs_to)) if belongs_to[i] == index]
                prototype = np.mean(dataset[instances_close], axis=0)
                # prototype = dataset[np.random.randint(0, num_instances, size=1)[0]]

                tmp_prototypes[index, :] = prototype

            prototypes = tmp_prototypes

            history_centroids.append(tmp_prototypes)

        #self.plot(dataset, history_centroids, belongs_to)

        return prototypes, history_centroids, belongs_to

    def execute(self, graph=True):
        dataset = self.load_dataset(self.data_as_txt)
        centroids, history_centroids, belongs_to = self.kmeans(self.k)
        if(graph):
            self.plot(dataset, history_centroids, belongs_to)

        return centroids


class Kpp(Kmeans):

    def initialization(self, dataset):
        X = dataset
        print(X)

        C = [X[0]]
        print(C)
        for k in range(1, self.k):
            #  find shortest distantce squared to center
            D2 = scipy.array([min([scipy.inner(c-x, c-x) for c in C]) for x in X])
            probs = D2/D2.sum()
            cumprobs = probs.cumsum()
            print(cumprobs)
            r = scipy.rand()
            for j, p in enumerate(cumprobs):
                if r < p:
                    i = j
                    break
            C.append(X[i])
        return np.array(C)

if __name__ == "__main__":
    kpp_means = Kpp("pts2.txt", 4)
    #centroids = kpp_means.execute(graph=True)

    kmeans = Kmeans("pts2.txt", 4)
    #centroids = kmeans.execute(graph=True)

    all_kmeans_centroids = []
    all_kpp_means_centroids = []

    for _ in range(1, 1000):
        kmeans_centroids = kmeans.execute(graph=False)
        all_kmeans_centroids.append(kmeans_centroids)

    for _ in range(1, 1000):
        kpp_centroids = kpp_means.execute(graph=False)
        all_kpp_means_centroids.append(kpp_centroids)

    with open("KmeansCentroids.txt", mode='w') as File:
        for i in all_kmeans_centroids:
            for j in i:
                File.write("{} {} ".format(j[0], j[1]))
            File.write("\n")

    with open("KppCentroids.txt", mode='w') as File:
        for i in all_kpp_means_centroids:
            for j in i:
                File.write("{} {} ".format(j[0], j[1]))
            File.write("\n")
\end{minted}

% ------
% R code
% ------
\subsection {Centroid Plots and Distribution of SSE}
\begin{minted}{r}
# --------------
# Plot centroids
# --------------
points <- read.table("pts2.txt")
kmeansCentroids <- read.csv("KmeansCentroids.txt", sep=" ", header=FALSE, na.strings="nan")[1:8]
kppCentroids <- read.csv("KppCentroids.txt", sep=" ", header=FALSE, na.strings="nan")[1:8]

plot(points, col=rgb(0,0,0,0.5))
for(i in 1:4) {
  points(x=kmeansCentroids[,1*i], y=kmeansCentroids[,2*i], pch=21, col=i+1, bg=rgb(0.1,0.1,0.1,0.1))
}
plot(points, col=rgb(0,0,0,0.5))
for(i in 1:4) {
  points(x=kppCentroids[,1*i], y=kppCentroids[,2*i], pch=25, col=i+1, bg=rgb(0.1,0.1,0.1,0.1))

# ------------------------
# Show distribution of SSE
# ------------------------
x <- read.csv("kmeans_sse.csv", sep=" ", header=FALSE)
x <- unlist(x, use.names=FALSE)
hist(x, main="Frequency of Sum of Mean Squared\n Errors in 100 k means trials", xlab="Sum of Mean Squared Errors", breaks=100)
mu <- mean(x)
variance <- var(x)
print(mu)
print(variance)
print(IQR(x))

x <- read.csv("kpp_sse.csv", sep=" ", header=FALSE)
x <- unlist(x, use.names=FALSE)
hist(x, main="Frequency of Sum of Mean Squared\n Errors in 100 k++ trials", xlab="Sum of Mean Squared Errors", breaks=100)
mu <- mean(x)
variance <- var(x)
print(mu)
print(variance)
print(IQR(x))
\end{minted}
