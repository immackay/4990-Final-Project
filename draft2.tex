% ----------------------------
% Ian MacKay and Oamar Kanji
% Clustering and Linear Models
% MATH 4990 Final Project
% ----------------------------
\documentclass[12pt]{article}
\usepackage{color,graphicx}
\usepackage{url}

\setlength{\oddsidemargin}{0in}
\setlength{\evensidemargin}{0in}
\setlength{\textwidth}{6.5in}
\setlength{\textheight}{9in}

% -----
% Start
% -----
\begin{document}
\title{\vspace{-1.5in} Time Series Models and Object Clustering}
\author{Oamar Kanji \& Ian MacKay}
\maketitle

\tableofcontents

\section{Introduction}
  \begin{center}

  \end{center}

\pagebreak

% ----------
% Clustering
% ----------
\section{Clustering}
  Clustering is the act of separating data into discrete groups to help analysis and prediction.
\subsection{k-means}
\subsection{k++means}
\subsection{Initialization Effects}
\subsection{Comparisons}
\subsection{Image Segmentation Experiments}

\pagebreak

% Time Series Models
\section{Time Series Models}
  Time Series Models are used for a number of analytical and predictive purposes, such as modeling fluctuating inventory levels, commodity prices, and stock prices.
\subsection{Box-Jenkins Methodology}
  A time series can contain any of the following components:
  \begin{itemize}
    \item Trend
    \item Seasonality
    \item Cyclic
    \item Random
  \end{itemize}
  % change these
  \begin{enumerate}
    \item Condition data and select a model
    \begin{itemize}
      \item Identify and account for any trends or seasonality in the time series.
      \item Examine the remaining time series and determine a suitable model.
    \end{itemize}
    \item Estimate the model parameters
    \item Assess the model and return to step one if necessary
  \end{enumerate}

\subsection{ARIMA Models}
  ARIMA Models are a combination of Autoregression, Integration, and Moving Average models. They are denoted by ARIMA(p,d,q), where p=autoregression factor, d=level of integration, and q=moving average factor

\subsection{ARIMA Variable Selection}

% Not sure about these next ones
% \subsection{Recurrent Neural Network}
% \subsection{Multivariate Analysis}
% \subsection{Price Prediction Experiments}

\pagebrea

% End stuff
\section{Conclusion}

\section{Figures \& Tables}

% Bibliography
\bibliographystyle{plain}
\bibliography{sources}

\section{Appendix}
% R Code
% Python Code
% Link to repo
\end{document}
